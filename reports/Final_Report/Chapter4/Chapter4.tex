% !TEX root =  ../Dissertation.tex

\chapter{System Requirements} \label{ch:system requirements}

\todo[inline]{Should I include personas?} The requirements outlined in this
section, provide a target for the CLI tools to meet, in order to be useful to
its users. This outline ensures that the CLI tools are easy to install by any
developer, they need minimal extra knowledge to use the tool. The interface
should be intuitive with minimal disruption to workflow. Also, the tools must
work in a variety of environments without issues.

\section{Agda Tree}

The Agda Tree CLI tool is an application that will run on the terminal. It will
extract and save the definition dependency graph from an Agda project, then it
will give the user commands to query and explore that graph. This tool has to
work with any Agda project and the user should be able to install the tool and
its dependencies easily.

\subsection{Functional Requirements}

\begin{minipage}{\linewidth}

The functional requirements are the features that the tool must implement to be
usable and meet the expectations of its users. For the tool to be easy to use,
it must be able to automatically extract the dependency graph from any Agda
project with little input from its user. The user is able to query the
dependency graphs and the output of the queries is intuitive to
understand.

\begin{table}[H]
\centering
\caption{Agda Tree Functional Requirements}
\label{tbl:Agda Tree Functional Requirements}
\begin{tblr}{
        colspec={|X[1]|X[6]|X[8]|}, hlines,
    }
ID & Name                           & Description                                                                                                                \\ 
1  & Definition Dependency Graph Extraction    & Extracts the definition dependency graph from an Agda Project                                              \\ 
2  & Querying the Definition Graph  & Allows the users to query the dependency graph and retrieve information. (See \cref{tbl:Definition Graph Queries} for queries)  \\ 
3  & Command-Line Interface         & User-friendly CLI that queries the dependency graph  \\ 
4  & Input Validation               & Validates user input and provides clear error messages for invalid inputs                                       \\ 
5  & Integration with Agda Projects & Agda projects are structured differently, all valid structures are supported                                      \\ 
6  & Output Generation              & Displays the query results in a readable format that follows the style of other Unix CLI tools                 \\ 
7  & Module Dependency Graph Extraction    & Extracts the module dependency graph from an Agda Project                                              \\ 
8  & Querying the Module Graph  & Allows the users to query the dependency graph to retrieve information. (See \cref{tbl:Module Graph Queries} for queries)  \\ 
\end{tblr}
\end{table}
\end{minipage}

\noindent
\begin{minipage}{\linewidth}

Martin Escardo asked in Mathstodon \cite{mathstodon} for possible queries that
this tool should implement. The tool queries both the definition and module
dependency graphs, the queries that can be made on the definition graph are the
following: 


\begin{table}[H]
\centering
\caption{Agda Tree Definition Queries}
\label{tbl:Definition Graph Queries}
\begin{tblr}{
        colspec={|X[1]|X[6]|X[8]|}, hlines,
    }
ID & Name                & Description                                                                                       \\ 
1  & Dependencies        & Get the direct or indirect dependencies of a definition        \\ 
2  & Dependents          & Get the direct or indirect dependents of a definition     \\ 
3  & Leafs               & Gets the leaves of the dependency graph, which would be the definitions that have no dependencies  \\ 
4  & Module Dependencies & Gets the direct or indirect modules that a definition uses                              \\ 
5  & Module Dependants   & Gets the direct or indirectmodules that use a definition                               \\ 
6  & Path to Leaf        & The longest path from a definition to any leaf                                                    \\ 
7  & Module Path to Leaf & The longest path from a definition to any leaf but only following the modules of the path         \\ 
8  & Roots               & The definitions with no dependents, meaning they aren't used anywhere                             \\ 
9  & Definition Type     & The definitions used for the type of the definition                                               \\ 
10 & Use count           & Counts how many times a definition is used                                                        \\
11 & Cycles              & Returns the cycles in the graph                                                                   \\
12 & Save Tree           & Saves the tree into a dot file                                                                    \\
13 & Path Between       & Finds the longest path between two definitions                                                    \\ 
\end{tblr}
\end{table}
\end{minipage}

\noindent
\begin{minipage}{\linewidth}

The queries that can be made on the module graph are mostly the same as above
except the module graph is a directed acyclic graph (DAG) giving it some
special properties. The queries are the following:

\begin{table}[H]
    \centering
    \caption{Agda Tree Module Queries}
    \label{tbl:Module Graph Queries}
    \begin{tblr}{
            colspec={|X[1]|X[6]|X[8]|}, hlines,
        }
        ID & Name             & Description                                                                                   \\ 
        1  & Dependencies     & Get the direct or indirect dependencies of a module             \\ 
        2  & Dependents       & Get the direct or indirect dependents of a module         \\ 
        3  & Leafs            & Gets the leaves of the dependency graph, which would be the modules that have no dependencies  \\ 
        4  & Path to Leaf     & The longest path from a module to any leaf                                                    \\ 
        5  & Roots            & The modules with no dependents, meaning they aren't used anywhere                             \\ 
        6  & Use count        & Counts how many times a module is used                                                        \\ 
        7  & Level Sort       & Returns a list of modules sorted by how far away it is from a leaf                            \\ 
        8  & Path Between     & Finds the longest path between two modules                                                    \\ 
        9  & Topological Sort & Returns a list of modules sorted topologically                                                \\
    \end{tblr}
\end{table}
\end{minipage}

\subsection{Non-Functional Requirements}

\begin{minipage}{\linewidth}

Since Agda Tree is a tool that slots into the workflow of developers, it must
be easy to use and performant. The tool integrates seamlessly and works on a
variety of projects regardless of size. To not disrupt the developer, the
queries must respond in less than 1 second. Agda is used mainly in Unix-based
systems, so this tool must be compatible with macOS and popular Linux
distributions.

\begin{table}[H]
    \centering
    \caption{Agda Tree Non-Functional Requirements}
    \label{tbl:Agda Tree Non-Functional Requirements}
    \begin{tblr}{
            colspec={|X[1]|X[6]|X[8]|}, hlines,
        }
        ID & Name                   & Description                                                                                                 \\ 
        1  & Extraction Performance & Extracts the dependency graph in 10 minutes depending on the size of the project               \\ 
        2  & Query Performance      & Responds to a query in under 1 second \\ 
        3  & Scalability            & Allows for fast querying of large projects\\ 
        4  & Usability              & Easy to use, with intuitive commands, clear documentation, and meaningful error messages.  \\ 
        5  & Compatibility          & Works on macOS and Linux \\ 
        6  & Reliability            & Handles bad inputs gracefully \\ 
        7  & Maintainability        & Well documented and well-structured to allow for new queries                         \\ 
        8  & Testability            & Tested to ensure queries give the correct output \\
    \end{tblr}
\end{table}
\end{minipage}

\section{Agda Comp}

The Agda Comp CLI tool, is an application that runs on the terminal. It
extracts the module dependency graph from an Agda project, produces the order
in which the modules should be type-checked and proceeds to compile the project
with that order.

\subsection{Functional Requirements}

\begin{minipage}{\linewidth}

The functional requirements for Agda comp are the features that the users need,
to compile their projects with minimal hassle. This tool works automatically,
the user only needs to input what they want to compile. It will create index
files and a make file that will compile the Agda Project based on the selected
strategy. This compilation must be safe and correct, it must compile all
necessary modules, and it shouldn't compile  a module and its dependencies
concurrently.

\begin{table}[H]
\centering
\caption{Agda Comp Functional Requirements}
\label{tbl:Agda Comp Functional Requirements}
\begin{tblr}{
        colspec={|X[1]|X[6]|X[8]|}, hlines,
    }
        ID & Name                           & Description                                                                                                                \\ 
        1  & Module Dependency Graph Parser & Parses Agda's dot file module dependency graph \\ 
        2  & Compilation Strategies         & Selection of compilation strategies to apply  \\ 
        3  & Compilation Customization      & Customization of parameters for the compilation strategy (i.e. amount of cores used) \\ 
        4  & Compilation                    & Creates index files and a make file that will safely compile all modules \\
        5  & Command-Line Interface         & User-friendly CLI with commands to run and customize compilation\\ 
        6  & Input Validation               & Validates user input and provides clear error messages for invalid inputs                                       \\ 
        7  & Integration with Agda Projects & Agda projects are structured differently and are all supported \\ 
        % 8  & Output Generation              & The CLI must output the compilation strategy into index files and a make file for the user                 \\ 
        8  & Speed Up Compilation            & Compiles Agda projects faster than a normal compilation.
\end{tblr}
\end{table}
\end{minipage}

\section{Non-Functional Requirements}

\begin{minipage}{\linewidth}

Agda Comp is a tool that slots in next to Agda seamlessly, so users who already
have Agda do not need extra setup. The tool is easy to use and works with any
Agda project. Agda is not developed for Windows, so the main focus is for the
tool to work in a Linux and macOS environment.

\begin{table}[H]
    \centering
    \caption{Agda Tree Non-Functional Requirements}
    \label{tbl:Agda Comp Non-Functional Requirements}
    \begin{tblr}{
            colspec={|X[1]|X[6]|X[8]|}, hlines,
        }
        ID & Name                   & Description                                                                                                 \\ 
        3  & Scalability            & Allows for compilation of large projects                                                   \\ 
        4  & Usability              & Easy to use, with intuitive commands, clear documentation, and meaningful error messages.  \\ 
        5  & Compatibility          & Works in macOS and Linux                                                                \\ 
        7  & Maintainability        & The codebase is well documented and well-structured to allow for new strategies                         \\ 
        8  & Testability            & The compilation strategies are tested to ensure the correctness and safety \\
    \end{tblr}
\end{table}
\end{minipage}

\section{Conclusion}

The functional requirements for Agda Tree ensure that it can be widely used
with little setup and outputs useful information to the user. All the queries
an Agda developer would need are included or simple to add. The non-functional
requirements define how the Agda Tree must behave, for it to slot into a
developer's workflow without issues.

The functional requirements for Agda Comp show the features needed for the
tool, it must work with any Agda project and give a choice on how to
compilation given some parameters. The non-functional requirements define the
behaviour of Agda Comp, it must work with any Agda project regardless of size
and the compilation strategies should be correct and safe. That is the
compilation strategies should compile every module necessary and do so without
compiling a module and its dependencies at the same time. Since the time it takes
to compile varies on the project size, performance measures aren't going to be
made.


% This section should detail your understanding of what you are planning to
% create. The section should aim to break down the overarching aims of the work into clear,
% measurable requirements that can be used in the evaluation of the project. This is why we often
% function of functional and non-functional requirements.
%
% \begin{itemize}
% \item Explain what the project should do and how it should do it 
% \item Functional requirements
% \begin{itemize}
%     \item Create graph from agda project
%     \item Should allow for asked queries (the ones mentioned in the Mathodon thread)
%     \item Reduce compilation time by 10\%? 
%     \end{itemize} 
% \item Non-Functional requirements
%     \begin{itemize}
%     \item Installation shouldn't be difficult
%     \item Queries should be done in under 2 seconds 
%     \item Creation of the graph shouldn't take more than 10 minutes 
%     \end{itemize}
% \enI ud{itemize}

