% !TEX root =  ../Dissertation.tex

\chapter{Introduction}

% The introduction should provide content for the report, discuss relevant background
% material, identify stakeholders, and state the aim(s) of the work. About 10\% of
% the report.

\section{Background}

When you first encounter a massive code base, containing years of development
from different people all adding their own code with their own style and
thought process, it can be overwhelming and difficult to parse through. On most
codebases you can ignore the fundamental functions that underpin a given
library, but in Agda it is quite useful to have knowledge of the underlying
definitions of the project. 

Agda is a functional programming language that can be used as a proof
assistant, following the propositions as types logic system \cite{agda_docs}.
Since proofs in Agda can be created almost entirely from scratch, defining the
most basic and fundamental types and operations it means that projects will
quickly balloon in size. While most of the time it is possible to use higher
level definitions when developing a proof, knowing the how the used types are
fundamentally defined can be quite beneficial. 

\section{Motivation}

Due to the size of these projects and how complex type definitions can be it
becomes difficult to fully grasp what the relationships between all the
definitions are. This creates the need for a tool that can analyze Agda
projects and give the user an easy to use interface to query the relationships
between definitions, in order to better understand the proofs. A tool that
would allow the user to explore these relationships would be of great benefit.

% This tool would also be allow the owner of the project to refactor their
% modules and change the structure of the project based on what the relationship
% between the definition are. For example if a definition isn't used often, it
% can be remove, if a definition depends on many other definitions it could be
% split into smaller pieces. This information would give the owner a better idea
% on where to put their focus.

Also, large Agda projects tend to deal with long compilation times. Agda is a
proof assistant, during "compilation", it type checks the entire project which
can take a 5 minutes or more depending on the computer and project. Agda will
keep track of what modules have already been compiled, so working on high level
modules isn't an issue as Agda will only type check files that have changed in
the project and high level modules don't have dependents that would need to be
re-checked. In contrast, low level files with many dependents, such as a module
defining the natural numbers, will cause a significant portion of the project
to re-compile. This effect will be particularly pronounced during refactoring
where significant changes will be made to the structure of the project, causing
Agda to re-check the whole project making most of the time spent refactoring,
spent waiting. 

Agda type checks the project sequentially, going through the dependencies of
each modules, this is done to mantain safety as a module won't be compiled
until its dependencies have. A tool that analyzes the relationship between
modules and finds modules that are safe to compile in parallel could lead to
significant time savings.

% Another feature for this tool would be decreasing compilation time, Agda type
% checks the project sequentially, one module at a time following the
% dependencies until it is done. This is done to avoid conflicts, a module will
% only be type checked once its dependencies are also type checked. But with this
% tool, the dependency graph is known so it is possible to parallelize the type
% checking. By type checking in parallel modules that don't depend on each other,
% it would more evenly distribute the work between CPU cores reducing overall
% type checking time.

    % \begin{itemize}
    % \item Explain how Agda projects can be massive and difficult to explore 
    % \item Explain how slow compilation lead to poor dev experience 
    % \end{itemize} 


\section{Problem Statement}

This project aims to create a CLI tool that will analyze an Agda project and
create a dependency graph from its definitions. Agda doesn't provide a built-in
tool to extract the definitions from a project, but there is an s-expression
extractor Agda version by Andrej Bauer \cite{andrej} that will allows a better
view into Agda's internal representation. With this graph the CLI will have
commands that allow the user to explore and query this graph, in order to gain
insight into the Agda project. As there aren't currently any tools that allows
you to query the definitions of an Agda project.

The CLI tool will also get the module dependency graph and generate Makefiles
and index files that will optimize the order of the modules compiled to
minimize compilation time.

\section{Objectives}

\begin{itemize}
    \item Getting defintions from s-expressions 
    \item Create dependency graph from definitions 
    \item Querying graph
\end{itemize}

\section{Project Structure}

\begin{itemize}
\item How to get s-expressions from agda 
\item How to get defintions from s-expressions 
\item How to get relationships from definitions 
\item How to create graph 
\item How to create queries 
\item Methods to improve compilation times 
\item Results
\end{itemize}

% Overview
%     \begin{itemize}
%     \item Overview of Agda and s-expressions 
%     \item Main goal is to represent an agda project as a graph for analysis 
%     \end{itemize} 


% Problem Statement
%     \begin{itemize}
%     \item Lack of a tool that allows for querying dependency information in
%         Agda projects 
%     \end{itemize} 

% \section{Use cases and future}
%
% Importants
%     \begin{itemize}
%     \item How the tool can be expanded and used
%     \end{itemize}
