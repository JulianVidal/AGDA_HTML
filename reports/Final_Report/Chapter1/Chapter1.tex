% !TEX root =  ../Dissertation.tex

\chapter{Introduction}

% The introduction should provide content for the report, discuss relevant background
% material, identify stakeholders, and state the aim(s) of the work. About 10\% of
% the report.
\marginpar{Add more citations to articles and different sources}
\marginpar{Add section about s-expressions}

\section{Background}

When you first encounter a massive code base, containing years of development
from different people all adding their own code with their own style and
thought process, it can be overwhelming and difficult to parse through. On most
codebases you can ignore the fundamental functions that underpin a given
library, but in Agda it is quite useful to have knowledge of the underlying
definitions of the project. 

Agda is a functional programming language that can be used as a proof
assistant, following the propositions as types logic system \cite{agda_docs}.
Since proofs in Agda can be created almost entirely from scratch, defining the
most basic and fundamental types and operations it means that projects will
quickly balloon in size. While most of the time it is possible to use higher
level definitions when developing a proof, knowing the how the used types are
fundamentally defined can be quite beneficial. 

\section{Motivation}

Due to the size of these projects and how complex type definitions can be it
becomes difficult to fully grasp what the relationships between all the
definitions are. This creates the need for a tool that can analyze Agda
projects and give the user an easy to use interface to query the relationships
between definitions, in order to better understand the proofs. A tool that
would allow the user to explore these relationships would be of great benefit.

% This tool would also be allow the owner of the project to refactor their
% modules and change the structure of the project based on what the relationship
% between the definition are. For example if a definition isn't used often, it
% can be remove, if a definition depends on many other definitions it could be
% split into smaller pieces. This information would give the owner a better idea
% on where to put their focus.

Also, large Agda projects tend to deal with long compilation times. Agda is a
proof assistant, during "compilation", it type checks the entire project which
can take a 5 minutes or more depending on the computer and project. Agda will
keep track of what modules have already been compiled, so working on high level
modules isn't an issue as Agda will only type check files that have changed in
the project and high level modules don't have dependents that would need to be
re-checked. In contrast, low level files with many dependents, such as a module
defining the natural numbers, will cause a significant portion of the project
to re-compile. This effect will be particularly pronounced during refactoring
where significant changes will be made to the bottom of the project, causing
Agda to re-check the whole project making most of the time spent refactoring,
spent waiting. 

Agda type checks the project sequentially, going through the dependencies of
each modules, this is done to mantain safety as a module won't be compiled
until its dependencies have. A tool that analyzes the relationship between
modules and finds modules that are safe to compile in parallel could lead to
significant time savings.

% Another feature for this tool would be decreasing compilation time, Agda type
% checks the project sequentially, one module at a time following the
% dependencies until it is done. This is done to avoid conflicts, a module will
% only be type checked once its dependencies are also type checked. But with this
% tool, the dependency graph is known so it is possible to parallelize the type
% checking. By type checking in parallel modules that don't depend on each other,
% it would more evenly distribute the work between CPU cores reducing overall
% type checking time.

    % \begin{itemize}
    % \item Explain how Agda projects can be massive and difficult to explore 
    % \item Explain how slow compilation lead to poor dev experience 
    % \end{itemize} 


\section{Problem Statement}

Agda doesn't provide a built-in tool to extract the definitions from a project,
but there is an s-expression extractor Agda version by Andrej Bauer
\cite{andrej} that will allows a better view into Agda's internal
representation. S-expressions aren't well suited data structure for the queries
on definition relationships, mostly contain syntax information that can be
discarded and each module has an s-expressions file. A data structure that
lends itself well towards queries of relationship between definitions, that
only contains information about the definitions and stores all the definitions
from a project, such as a graph. A tool that could explore the dependency graph
of the definitions, would allow for a faster understanding of a large code
base.

While Agda doesn't have a tool to extract definitions from projects, it does
have a tool to create a dependency graph from the modules. For compilation time
improvement, only the modules are of interest as such an effective
representation is already available. The challenge is what is the most
effective way to traverse a dependency graph, concurrently, while mantaining
safety (i.e. don't compile modules and its dependencies at the same time). 


\section{Objectives}

This objective of this project is to create a dependency graph of defintions
from an Agda project by using the s-expression representation of modules using
the s-expression extractor \cite{andrej}. Then create build a (Command Line
Interface) CLI that will create the dependency graph from any Agda project and
give the user access to commands that query the dependency graph. These queries
will let the user explore the graph, in order to gain insight into the Agda
project. As there aren't currently any tools that allows you to query the
definitions of an Agda project. 

The objective is also to build another CLI tool that will create the module
dependency graph from an Agda project and generate the makefile with the
optimal order that modules should be compiled to reduce compile time.

\section{Project Structure}

First, there will be an explanation of what s-expressions are and how they are
being read. These s-expressions use tags to name different symbols in a module,
only the tags that contain definitions are important. Then, show how the
s-expressions are structured, the way that defintions can be found and getting
their dependencies. Using the information gained for the s-expressions, they
will be turned into a graph that can be easily queried for relationships. Some
example queries will be demonstrated, to underline their utility and
implementation.

Secondly, by analyzing different algorithms for traversing a module dependency
graph there will be an exploration on how parallelism can improve compile time
of an Agda project. Results from different algorithms and projects will be
shown.
