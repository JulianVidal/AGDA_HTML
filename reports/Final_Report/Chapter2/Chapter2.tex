% !TEX root =  ../Dissertation.tex

\chapter{Literature Review}

\marginpar{Add literature about code decay}

There are many ways to represent code using graphs, each with their own utility
and purpose. There are Abstract Syntax Tree (AST), Control Flow Graph (CFG),
Data Flow Graph (DFG), Program Dependence Graph (PDG), and Code Property Graph
(CPG). These represenations encode all the behaviour and properties of a
project, they can have many uses such as for code vulnerability detection
\cite{graph_for_code_vuln}. These representation also allows for static
analysis, where code isn't ran but the structure is analyzed for software
validation \cite{static_analysis}. For this project, these graphs encode far
more information that is require, as for the dependency graph only the
relationship between definitions is of value. 


\marginpar{Add section explaining dependency graph}

\section{Related works}

The Language Server Protocol (LSP) is used by IDEs, such as Vistual Studio Code
and Intellij, to provide features like goto defintion and goto references. The
LSP sits between the code editor and the language server, it is the language
server that analyzes the code structure to support the features
\cite{LSP_implementation}. Language server has the functionality to implement
this project, but, they are meant to be used while editing code and the
language server isn't made to be used with custom queries. Examples of LSP's
are Jedi \cite{jedi_lsp} and Agda's langauge server \cite{agda_lsp}, while they
contain the functionality needed they are only meant to work on the current
open file, not a whole project.

Ctags \cite{ctags} is a tool that indexes all the symbols in a project, this is
helpful to get all the definitions from an Agda project. But it doesn't capture
the relationship between the symbols it finds.

Graph Buddy is an interactive code dependency browsing and visualization tool
\cite{graph_buddy}. It takes large Java codebases turns them into Semantic Code
Graphs (SCG), and creates a visualization of this graph. This graphs shows
dependencies between modules, classes and methods which helps the developer
better understand the project and tackle the pervasive issue of code decay
\cite{code_decay_evidence}. Graph Buddy is integrated as to an IDE as a plugin,
where the user can seamlessly explroe the visualization. 

There is also a tool \cite{call_graph_vis} that visualizes the call graph of a
coding project, this allows for better developer experience. It works in
real-time, while the user is editing the code the graph will automatically
update to show the changes.

\section{Parallel Compilation}

\marginpar{Add section explaining DAGs}
Type-checking is a computationally expensive task that hinders the workflow of
a developer, this has lead to work to parallelize type-checking algorithms or
make them incremental. Parallel type-checking aims to type check different
parts of a project at the same time, while incremental type-checking aims to
allow the developer to type check a small change in the project without having
to type-check the whole project again.

An example is the work by Newton et al \cite{paralele_comp_haskell} which seeks to
parallelize type-checking with Haskell. Also, the work by Zwaan et al.
\cite{incremental_type_checking} using scope graphs for incremental
type-checking.

However, this project doesn't aim to optimize the type checking algorithm
themselves, rather, find independent modules that can be type checked together.
This is aligns more close with the following paper that explores reducing FPGA
compile time by changing from a monolithic compilation style to compiling
separate blocks in parallel \cite{FPGA}.

\marginpar{Add section explaining scheduling problem}
Fundamentally the compilation problem in this project is a scheduling problem,
which is NP-complete\cite{scheduling}. This means that there many algorithms
that attempt to tackle this problem by following different assumptions as shown
by Yves Robert \cite{scheduling}.

\marginpar{Add section explaining s-expressions}

\section{Conclusion}

There are many tools that analyze and visualize the overall structure of
programming projects. Allow for static code analysis, where software can be
validated and developers can have an easier time exploring a project. However,
most of the tools are not easy to query by the user, meant to be used while
editing a file and are reserved towards more popular languages like Java. A
tool that is able to read an entire Agda project and gives the user access to
the underlying graph is still missing.

Slow compilers is a common problem, hurting developer experience. Due to the
monolithic nature of compilers, parallelization becomes a route to follow when
optimizing compilation time. While parallelization can be applied to the
type-checking algorithm itself, this project looks to type check modules in
parallel while the type checking algorithm remains the same. This problem is
closer to a scheduling problem, where the goal is to find the optimal way to
assign tasks to multiple machines to reduce completion time. There is no tool
in Agda that attempts to apply an scheduling algorithm to the type-checking of
modules, which could lead to significant speed ups.  


% This is often referred to the ‘literature review’ section. It is one of the most important
% section of the Project Proposal and the Project Report. It is where you demonstrate that you
% understand the state-of-the-art in the field you’re working. Towards the end of this sections it’s
% normally a good idea to explain how your aim / work / idea / contribution differs from the nearest
% work in the field.


% Research papers to explore:
% \begin{itemize}
%     \item State of the art in dependency graph extraction 
%     \item state of the art in dependency graph exploration
%     \item State of the art in Agda compilation 
% \end{itemize}
%
% \begin{itemize}
% \item s-expressions 
% \item How graphs can be used to represent the definitions
% \item Most tools focus on data-flow for static analysis, I am looking
%     for a simpler tools that allows the querying of the relationships between
%     definitions.
% \end{itemize}
%
% Tools that do a similar job
% \begin{itemize}
% \item \href{https://github.com/universal-ctags/ctags}{CTags}
%     \begin{itemize}
%     \item CTags indexes all the symbols of a project
%     \item This works only on definitions of functions, not where they are referenced
%     \end{itemize}
% \item \href{https://jedi.readthedocs.io/}{Jedi - an awesome autocompletion, static analysis and refactoring library for Python}
%     \begin{itemize}
%     \item It finds function definitions and references which is what we are looking
%         for
%     \end{itemize}
% \item \href{https://github.com/agda/agda-language-server/tree/master}{Agda-Language-Server}
%     \begin{itemize}
%     \item An LSP for agda, it parses the files and analysis them, similar to what we
%         want but it doesn't seem to index them
%     \end{itemize}
% \item \href{https://cormack.uwaterloo.ca/~olhotak/pubs/ecoop12.pdf}{Application-only Call Graph Construction }
%     \begin{itemize}
%     \item A call graph represents the relationships between functions
%     \end{itemize}
% \item \href{https://dl.acm.org/doi/pdf/10.1145/199448.199462}{Precise Interprocedural Dataflow Analysis via Graph Reachabilit}
%     \begin{itemize}
%     \item Data flow is a way to statically analyse code before running it, this is
%         far more complex than what I want.
%     \end{itemize}
% \item \href{https://dl.acm.org/doi/abs/10.1145/3583660.3583664}{CFL/Dyck Reachability: An Algorithmic Perspective}
%     \begin{itemize}
%     \item Analyses the decidability and complexity of this problem
%     \end{itemize}
% \end{itemize}
