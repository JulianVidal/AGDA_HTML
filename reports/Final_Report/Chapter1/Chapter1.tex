% !TEX root =  ../Dissertation.tex

\chapter{Introduction}

% The introduction should provide content for the report, discuss relevant background
% material, identify stakeholders, and state the aim(s) of the work. About 10\% of
% the report.

\section{Background}

Large codebases with years of development can be overwhelming to navigate
\cite{code_decay} and in Agda understanding the underlying definitions is
important for new proofs. Dependency graphs can help in navigating the code
base and identifying areas needing refactoring\cite{dep_grah_refactoring}.

Agda is a functional programming language and proof-assistant based
on propositions-as-types logic system \cite{agda_docs}. Proofs are made by
defining the most fundamental types and operations causing projects to rapidly
expand. When making a proof, knowing how the types are fundamentally defined is
helpful, but this information isn't readily available.

\section{Motivation}


Understanding relationships between definitions in large Agda projects is
challenging due to their complexity. A tool that analyses these relationships
and provides an intuitive interface can improve code comprehension and
refactoring, reducing code decay \cite{fowler2018refactoring}. For example,
unused definitions can be removed, and highly dependent modules split into
smaller parts. 

Additionally, large Agda projects face long compilation times due to sequential
type-checking of modules. Changes to high-level modules aren't a problem but
low-level modules when modified will re-compile all their dependents. This
issue worsens during refactoring, where structural changes lead to unnecessary
re-checks. 

Agda's sequential type-checking gives an opportunity for parallel compilation.
Modules without dependency conflicts can be compiled concurrently, potentially
reducing compilation time. A tool identifying safe modules for parallel
compilation could lead to time savings.

\section{Problem Statement}

Agda doesn't have native commands for extracting project definitions. However,
Andrej Bauer's extractor gives a view into Agda's internal representation as
s-expressions \cite{andrej}. This project aims to transform these s-expressions
into dependency graphs for the user to explore for better codebase navigation.

To compile modules safely a module dependency graph is need which Agda already
provides. The challenge is safely traversing the graph concurrently. Different
strategies have to be tested against well-known libraries to gauge their
effectiveness. 

\section{Objectives}

The objective of this project is to create a Command-Line Interface (CLI) that
generate the dependency graph from any Agda project using the s-expression
representation \cite{andrej}. Then the user has access to queries that lets
them explore the graph and gain insight into the Agda project. There aren't any
tools currently that have these features. 

The second objective is to build a CLI tool that reduces compilation time. By
automatically generating the module dependency graph from an Agda project.
Applying a strategy to find the optimal order that modules should be compiled
in parallel while maintaining safety. And lastly, running that strategy,
compiling the entire project.

\section{Project Structure}

First, there will be an exploration of related works pertaining to using graphs
for code analysis. As well as, an explanation of s-expressions, other instances
of parallel computation and describing the scheduling problem. Second, the
requirements and overall design of the CLI tools will be own, along with an
explanation of what strategic are going to be employed for compilation. Third,
an explanation of the implementation of the designed systems. Fourth, an
evaluation of the tools and the performance of the compilation strategies.
Lastly, an overview of the results, limitations and future improvements. The
project was uploaded to GitHub
\href{https://github.com/JulianVidal/AGDA_HTML}{here}  and to University of
Birmingham School of Computer Science GitLab
\href{https://git.cs.bham.ac.uk/projects-2024-25/jcv116}{here}.
