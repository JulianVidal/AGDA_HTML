% !TEX root =  ../Dissertation.tex

\chapter{System Requirements}

The CLI tools should be easy to install by any developer, the developer
shouldn't need any extra knowledge to use the tool. The interface must be
intuitive and there should be little friction between the development workflow
and the use these tools. Also, the tools must work in a variety of environments
without issues.


\section{Agda Tree}

The Agda Tree CLI tool, is an application that will run on the terminal. It
will extract and save the definition dependency graph from an Agda project,
then it will give the user commands to query and explore that graph. This tool
should work with any Agda project and the user should be able to install the
tool without any extra dependencies.


\subsection{Functional Requirements}

The functional requirements are the features that the tool must implement to be
usable and meet the expectation of users. For the tool to be easy to use, it
must be able to automatically extract the dependency graph from any Agda
project. The user should be able to easily query the dependency graphs and the
output of the queries should be intuitive to understand.

\begin{table}[H]
\centering
\caption{Agda Tree Functional Requirements}
\label{table:Agda Tree Functional Requirements}
\begin{tblr}{
        colspec={|X[1]|X[6]|X[8]|}, hlines,
    }
ID & Name                           & Description                                                                                                                \\ 
1  & Definition Dependency Graph Extraction    & The tool must parse Agda projects and construct a definition dependency graph                                              \\ 
2  & Querying the Definition Graph  & The CLI must allow the users to query the dependency graph to retrieve important information. (See table \ref{table:Definition Graph Queries} for queries)  \\ 
3  & Command-Line Interface         & The tool must provide a user-friendly CLI with commands for querying the dependency graph                                  \\ 
4  & Input Validation               & The CLI must validate user input and provide clear error messages for invalid inputs                                       \\ 
5  & Integration with Agda Projects & Agda projects are structured differently, this tool must support all valid structures                                      \\ 
6  & Output Generation              & The CLI must display the query results in a readable format that follows the style of other Unix CLI tools                 \\ 
7  & Module Dependency Graph Extraction    & The tool must parse Agda projects and construct a module dependency graph                                              \\ 
8  & Querying the Module Graph  & The CLI must allow the users to query the dependency graph to retrieve important information. (See table \ref{table:Module Graph Queries} for queries)  \\ 
\end{tblr}
\end{table}

Martin Escardo asked in Mathstodon for all the possible queries that this tool
should implement. This tool will allow for the querying of both the definition
and module dependency graphs, the queries that can be made on the definition
graph are the following: 

\begin{table}[H]
\centering
\caption{Agda Tree Definition Queries}
\label{table:Definition Graph Queries}
\begin{tblr}{
        colspec={|X[1]|X[6]|X[8]|}, hlines,
    }
ID & Name                & Description                                                                                       \\ 
1  & Dependencies        & Get the dependencies of a definition what definitions it uses both directly and indirectly        \\ 
2  & Dependents          & Get the dependents of a definition, where the definition is used both directly and indirectly     \\ 
3  & Leafs               & Gets the leafs of the dependency graph, which would be the definitions that have no dependencies  \\ 
4  & Module Dependencies & Gets the modules that a definition uses both directly and indirectly                              \\ 
5  & Module Dependants   & Gets the modules that uses a definition both directly and indirectly                              \\ 
6  & Path to Leaf        & The longest path from a definition to any leaf                                                    \\ 
7  & Module Path to Leaf & The longest path from a definition to any leaf but only following the modules of the path         \\ 
8  & Roots               & The definitions with no dependents, meaning they aren't used anywhere                             \\ 
9  & Definition Type     & The definitions used for the type of the definition                                                  \\ 
10 & Use count           & Counts how many times a definition is used                                                        \\
\end{tlbr}
\end{table}

The queries that can be made on the module are mostly the same the module graph
is a directed acyclic graph (DAG) giving it special properties. The queries are
the following:

\begin{table}[H]
    \centering
    \caption{Agda Tree Module Queries}
    \label{table:Module Graph Queries}
    \begin{tblr}{
            colspec={|X[1]|X[6]|X[8]|}, hlines,
        }
        ID & Name             & Description                                                                                   \\ 
        1  & Dependencies     & Get the dependencies of a module what modules it uses both directly and indirectly            \\ 
        2  & Dependents       & Get the dependents of a module, where the module is used both directly and indirectly         \\ 
        3  & Leafs            & Gets the leafs of the dependency graph, which would be the modules that have no dependencies  \\ 
        4  & Path to Leaf     & The longest path from a module to any leaf                                                    \\ 
        5  & Roots            & The modules with no dependents, meaning they aren't used anywhere                             \\ 
        6  & Use count        & Counts how many times a module is used                                                        \\ 
        7  & Level Sort       & Returns a list of modules sorted by how far away it is from a leaf                            \\ 
        8  & Path Between     & Finds the longest path between two modules                                                    \\ 
        9  & Topological Sort & Returns a list of modules sorted topologically                                                \\
    \end{tblr}
\end{table}

\subsection{Non-Functional Requirements}

Since Agda Tree is meant to be a tool that slots into the workflow of
developers, it is critical for it to be easy to use and performant. The tool be
plug and play, working on a variety of projects regardless of size. To not
disrupt the developer, the queries should respond quickly. Agda is used mainly
with Unix based systems, so this tool must be compatible with macOS and popular
Linux distributions.

\begin{table}[H]
    \centering
    \caption{Agda Tree Non-Functional Requirements}
    \label{table:Agda Tree Non-Functional Requirements}
    \begin{tblr}{
            colspec={|X[1]|X[6]|X[8]|}, hlines,
        }
        ID & Name                   & Description                                                                                                 \\ 
        1  & Extraction Performance & The tool must extract the dependency graph in 10 minutes depending on the size of the project               \\ 
        2  & Query Performance      & The tool must be able to respond to a query in under 2 seconds                                              \\ 
        3  & Scalability            & The tool should allow for fast querying of large projects                                                   \\ 
        4  & Usability              & The tool must be easy to use, with intuitive commands, clear documentation, and meaningful error messages.  \\ 
        5  & Compatibility          & The tool should work in any operating system                                                                \\ 
        6  & Reliability            & The tool should handle bad inputs gracefully                                                                \\ 
        7  & Maintainability         & The codebase should be well documented and well-structured to allow for new queries                         \\ 
        8  & Testability            & The queries should be tested to ensure the correct output is given                                          \\
    \end{tblr}
\end{table}


\section{Agda Comp}

The Agda Comp CLI too, is an application that will run on the terminal. It will
extract the module dependency graph from an Agda project and produce the order
in which the modules should be type-checked.

\subsection{Functional Requirements}

\section{Non-Functional Requirements}

% This section should detail your understanding of what you are planning to
% create. The section should aim to break down the overarching aims of the work into clear,
% measurable requirements that can be used in the evaluation of the project. This is why we often
% function of functional and non-functional requirements.
%
% \begin{itemize}
% \item Explain what the project should do and how it should do it 
% \item Functional requirements
% \begin{itemize}
%     \item Create graph from agda project
%     \item Should allow for asked queries (the ones mentioned in the Mathodon thread)
%     \item Reduce compilation time by 10\%? 
%     \end{itemize} 
% \item Non-Functional requirements
%     \begin{itemize}
%     \item Installation shouldn't be difficult
%     \item Queries should be done in under 2 seconds 
%     \item Creation of the graph shouldn't take more than 10 minutes 
%     \end{itemize}
% \enI ud{itemize}
