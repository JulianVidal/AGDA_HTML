% !TEX root =  ../Dissertation.tex

\chapter{Evaluation}

This section evaluates the Agda Tree and Agda Comp tools with respect to the
function and non-functional requirements set on \cref{ch:system requirements}. These requirements outline the purpose of these tools and their
expected behaviour.

\section{Agda Tree} \label{sec:eval agda tree}

Agda Tree extracts the definitions and relationships from an Agda project, and
uses Agda to generate the module dependency graph then imports them into
NetworkX. The queries outlined in \cref{tbl:Definition Graph Queries} for the
definition graph and in \cref{tbl:Module Graph Queries} for the module graph,
are implemented. These methods are exposed through a command line interface
that the user uses to execute those queries on their Agda projects. The python
library argparse validates the user input. Lastly, CLI the outputs the modules to
console, which can be piped to other utilities. The CLI will work
with any Agda project.

At first, HTML files that Agda generates were explored instead of
s-expressions. These HTML files display and format the code with colours and
links. Styling the text differently depending on if it was a definition,
keyword, type, or operator. Also, the definitions were hyperlinks that linked
back to the module that defined them. By exploring the HTML, a definition
dependency graph could be created. The main issue becomes parsing the HTML
files, finding which keywords fell into which definition is challenging.
Switching to using s-expressions which are easier to parse was a significant
step forward.

Clojure was considered instead of Python to store the graphs, as it is made to
store information as graphs and make queries. However, Clojure requires Java,
isn't popular, and it is more difficult to learn. This makes Python the better
choice.

Agda Tree the extracts s-expressions from an Agda project but depending on the
size of the project and the user's computer it can take more than 10 minutes as
seen on \cref{sub:agda tree performance}. Most queries are completed in under 2
seconds regardless of project size, but queries that require finding a path in
the graph can take far longer depending on the size of the graph. The CLI is
well-documented in the \texttt{README.md} file and the \texttt{--help} option,
making it easy to learn and understand. Since Agda is mainly developed on macOS
and Linux, the tools were tested in WSL and a MacBook Air without issues. If
the CLI is given bad inputs it returns an error from NetworkX explaining what
the issue is. The code base is unit tested allowing for better maintainability
and testability. Therefore, Agda Tree meets all its functional requirement but
doesn't fully meet its non-functional requirements due to performance. The
requirements as described on \cref{tbl:Agda Tree Functional Requirements} for
functional requirements and \cref{tbl:Agda Tree Non-Functional Requirements}
for non-functional requirements.

\subsection{Limitations}

A limitation of the tool is finding the maximum path between a node and a leaf,
depending on the node the query can run indefinitely. This limits the usage of
the query. Another limitation is the verbosity of the definition names and the
amount of definitions. The TypeTopology definition graph has more than 50,000
definitions but in further inspection not all these definitions provide
valuable information to the user. Sometimes they come from Agda's backend which
isn't valuable. Also, the definition names become difficult to understand due
to their length, and some definitions are ambiguous, so they require an ID
number. Overall making the name of the definitions cumbersome to use.


\section{Agda Comp} \label{sec:eval agda comp}

Agda Comp uses Agda to generate the module dependency graph and import it into
Python. With this graph, Agda Comp lets the user pick between multiple
strategies for compilation. The user chooses how many cores to use during
compilation. Then the necessary index files and make file are generated,
compiles the project and lastly cleans up the generated files. The CLI has a
\texttt{README.md} and a usage message with the \texttt{--help} option making
it user-friendly and easy to learn. The argparse library validates the user
input to the CLI and gives a user-friendly error message if not. The tool
detects the project structure from the user input, making
it usable in any project structure. Therefore, Agda Comp meets its
functional requirements, except it speeds up compilation for some Agda projects
which is explored in \cref{sub:eval comp strat}. The functional
requirements were described in \cref{tbl:Agda Comp Functional Requirements}.

Agda Comp works in any project, it has a simple command line interface for
choosing the compilation strategy. It was tested within WSL and macOS
environment, for compatibility. The codebase is well-documented, and the
strategies were tested for correctness and safety. Therefore, Agda Comp
meets all of its non-functional requirements as described on \cref{tbl:Agda
Comp Non-Functional Requirements}.

\subsection{Limitations} \label{sub:eval comp limitations}

This approach encounters two limitations, the first is that to create the
module dependency graph it has to compile the project. Which means that for
Agda Comp to save time the user only makes changes that don't alter the module
dependency graph, such that a previous dependency graph can be re-used.

The second limitation is the overhead caused by the loading of the interface
files. Interface files are the compiled Agda source code, when a module is type
checked the information is stored in an interface file and used when type checking
modules that depend on it. Every time a new process is created, it has to load
all the interface files it requires to type-check a module and discard them
once done, then the next process might load the same interface files again
wasting resources. Meanwhile, if everything was compiled in one process like
normal it would only load the interface files once. This suggests that while
parallelization could improve compilation time, it can't be done at the user
level calling the Agda Type Checker multiple times. The Agda Type Checker must
be parallelized from within, this would mean all the necessary interface files
can be loaded once and the type checker can check modules in parallel with less
overhead.

\subsection{Compilation Strategies} \label{sub:eval comp strat}

The \cref{tbl:WSL comp results} and \cref{tbl:martin comp results}
show the compilation times of each strategy. The compilation testing consisted
of 7 strategies across 3 libraries. The three libraries picked are TypeTopology
\cite{type-topology}, unimath \cite{agda-unimath} and Agda's stdlib
\cite{stdlib}. These libraries vary in size and methodology, for example
unimath structure is of many small independent modules while TypeTopology has
less but longer modules.

The 7 strategies tested are normal, which is the standard compilation that
gives a baseline for the other strategies. The unsafe test, which attempts to
compile all modules in 4 index files at the same time, without regards to
safety which shows the potential of parallelization. Then the level strategy
using method A to split modules into index files described in 
\cref{sub:imp lvl strategy}, the modules in the same level will be tested when
split into 2 index files or 5 index files. The next test is using method B
instead, where each index file has at most 2 modules or 5 modules. Lastly, the
disjoint strategy is tested which has no parameters.

% Tested on Agda 2.0.17, WSL has 16GB of ram with 8 performance cores at 5.4 GHz.
% Macos tested on Agda 2.0.17 has nGB of ram with n cores and n GHz.

\todo[inline]{Should I add the tables here, or move some to the appendix?}

\begin{table}[H]
  \centering
  \caption{Results from WSL Testing Compilation Strategies}
  \label{tbl:WSL comp results}
  \begin{tblr}{
      colspec={|Q[m]|Q[c, m]|Q[c, m]|Q[c, m]|Q[c, m]|Q[c, m]|Q[c, m]|Q[c, m]|}, hlines,
      cells   = {font = \fontsize{8pt}{10pt}\selectfont},
    }
    {seconds\\(\%)} & Normal      & Unsafe     & {Level A\\2 cores} & {Level A\\5 cores} & {Level B\\2 cap} & {Level B\\4 cap} & Disjoint    \\
    TypeTopology & {575\\(100\%)} & {280\\(49\%)} & {354\\(62\%)}        & {355\\(62\%)}        & {482\\(84\%)}      & {394\\(69\%)}      & {528\\(92\%)}  \\
    stdlib       & {289\\(100\%)} & {147\\(51\%)} & {265\\(92\%)}        & {243\\(84\%)}        & {309\\(107\%)}     & {261\\(90\%)}      & {289\\(100\%)} \\
    Unimath      & {459\\(100\%)} & {219\\(48\%)} & {862\\(188\%)}       & {362\\(79\%)}        & {717\\(156\%)}     & {644\\(140\%)}     & {462\\(101\%)} \\
  \end{tblr}
\end{table}

\begin{table}[H]
  \centering
  \caption{Results from Martin Escardo Testing Compilation Strategies Mac Mini}
  \label{tbl:martin comp results}
  \begin{tblr}{
      colspec={|Q[m]|Q[c, m]|Q[c, m]|Q[c, m]|Q[c, m]|Q[c, m]|Q[c, m]|Q[c, m]|}, hlines,
      cells   = {font = \fontsize{8pt}{10pt}\selectfont},
    }
    {seconds\\(\%)} & Normal      & Unsafe     & {Level A\\2 cores} & {Level A\\5 cores} & {Level B\\2 cap} & {Level B\\4 cap} & Disjoint    \\
    TypeTopology & {345\\(100\%)} & {172\\(50\%)} & {287\\(83\%)}        & {265\\(77\%)}        & {344\\(100\%)}     & {295\\(86\%)}      & {369\\(107\%)} \\
    stdlib       & {189\\(100\%)} & {123\\(65\%)} & {241\\(128\%)}       & {197\\(104\%)}       & {231\\(122\%)}     & {203\\(107\%)}     & {203\\(107\%)} \\
    Unimath      & {302\\(100\%)} & {168\\(56\%)} & {863\\(286\%)}       & {575\\(190\%)}       & {633\\(210\%)}     & {568\\(188\%)}     & {304\\(101\%)} \\
  \end{tblr}
\end{table}

\begin{table}[H]
  \centering
  \caption{Results from MacBook Air M4 Compilation Strategies }
  \label{tbl:mba comp results}
  \begin{tblr}{
      colspec={|Q[m]|Q[c, m]|Q[c, m]|Q[c, m]|Q[c, m]|Q[c, m]|Q[c, m]|Q[c, m]|}, hlines,
      cells   = {font = \fontsize{8pt}{10pt}\selectfont},
    }
    seconds (\%)                  & Normal      & Unsafe     & { Level A\\2 cores} & {Level A \\5 cores} & {Level B\\2 cap} & {Level B\\4 cap} & Disjoint    \\
    TypeTopology & {371\\(100\%)} & {234\\(63\%)} & {331\\(89\%)}        & {298\\(80\%)}        & {366\\(99\%)}      & {334\\(90\%)}      & {340\\(92\%)}    \\
    stdlib       & {191\\(100\%)} & {134\\(70\%)} & {316\\(165\%)}       & {267\\(140\%)}       & {277\\(145\%)}     & {273\\(143\%)}     & {223\\(117\%)}   \\
    Unimath      & {335\\(100\%)} & {295\\(88\%)} & {1437\\(429\%)}      & {802\\(239\%)}       & {674\\(201\%)}     & {687\\(205\%)}     & {314\\(94\%)}    \\
  \end{tblr}
\end{table}

The tables show compilation time is affected on the project structure. The
specs of the machines used can be seen in \cref{tbl:WSL specs},
\cref{tbl:martin specs} and \cref{tbl:mba specs}. TypeTopology benefitted the
most from the compilations strategies achieving about a 40\% faster compilation
in WSL, 23\% faster in the Mac Mini and 20\% faster in the MacBook Air.
Although the results from the remaining Agda projects aren't as promising, with
the macOS results showing little to no improvement in any safe strategy. The
results also show that the compilation time also depends on the user's system,
with WSL achieving a speed-up in level A 5 cores for all projects.

The cause in the difference in compilation improvement per project, is likely
due to the limitation discussed in the \cref{sub:eval comp limitations}. Since
unimath has many small modules, there is a large amount of interface files that
are constantly being loaded and unloaded when normal compilation can load all
the interface files at once and keep them. This is a hypothesis that could be
explored further.

\section{Conclusion}

Both Agda Tree and Agda Comp meets most of the functional and non-functional
requirements set out in \cref{ch:system requirements}. Agda Tree effectively
extracts definition dependency graphs from any Agda project, but it can take a
long time to generate that graph. All queries are implemented, however some
queries can take an indefinite amount of time to complete. One of the
limitations of Agda Tree is the difficult to understand definitions and
definition names, as the Agda backend doesn't make them user-friendly.

Agda Comp imports the module dependency graph from any Agda project and
compiles the project in parallel by using different strategies. Although, the
effectiveness of those strategies varies depending on the system and the Agda
project. There is a limitation where creating the module dependency graph
compiles the project. Meaning this tool is used when the user makes a change
that doesn't alter the module dependency graph, limiting its scope.


% \begin{itemize}
% \item Mention compilation time reduction, and improvements 
% \item Mention how some queries take too long but were all implemented 
% \item Mention how trees take some time but can be re-used instantaneously 
% \item Creates graph from any agda project
% \end{itemize}

% Usually you evaluate your project with regard to the functional and non-functional
% requirements you set out in the earlier chapter. This doesn’t necessary mean that your project was
% successful but, if these requirements were appropriately specified, you it’s likely that your project was
% successful. You might be reiterating some points from the Testing and Success Measurement
% chapter in your Project Report.

% \begin{itemize}
% \item Explain how agda tree meets the functional and non-functional requirements.
% \item Explain how having each module type check by itself is inneficient due to agai files
% \item Explain how you moved stuff to index to avoid this and different ways to space of the modules  
% \item Explain how conccurently compiling at each level isn't efficient as it
%   would be best to have massive modules with many dependencies be compiled next
%   to other massive modules with disjoint dependnecies. Instead of a couple of
%   modules at a time.
% \item Discuss limitation 
% \item where the tool might need improvement 
% \item Recap of how the CLI tool meets the research objectives 
% \item integraiont into ides
% \end{itemize}

% \begin{itemize}
% \item Explain how the strategies are giong to be tested 
% \item Explain what strategies will be tested and why 
% \item Explain what libraries will be used for testing 
% \item Show table with the results of each test
% \item Discuss limitation, constrints of scenarios 
% \item where the tool might need improvement 
% \item Recap of how the CLI tool meets the research objectives 
% \item Possible improvement havea Agda do type checking from inside and creae
%   module dependenc ygraph without type-checkiing
% \end{itemize}
