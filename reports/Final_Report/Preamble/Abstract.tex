
\chapter*{Abstract}
\addcontentsline{toc}{chapter}{Abstract}

When dealing with large Agda projects it becomes difficult to navigate the
project and slower to type-check. This paper addresses these issues with two
CLI tools. Agda Tree which constructs dependency graphs from Agda definitions
to help with code navigation and Agda Comp, which aims to optimize compilation
time by parallelizing the type-checker using module dependency graphs. Agda
Tree uses Andrej Bauer's s-expression extractor to generate the definition
dependency graph and Agda's built-in command to generate the module dependency
graph. The CLI lets developers query these relationships to better understand
codebase. Agda Comp applies two strategies, level sort and disjoint level, to
parallelize the module compilation while ensuring safety and correctness
according to the module dependency graph.

The evaluation shows that Agda Tree extracts dependency graphs from Agda
projects like TypeTopology, but the performance of the queries depends on the
size of the project. Agda Comp achieves compilation improvements of up to 38\%
depending on the user's system, and project size and structure. However, the
overhead of reloading Agda's interface files limits the gains in most
scenarios.


% Usually 100-300 words stating the salient points of the report. It should help your reader
% to decide whether the report is relevant to her or his interests.
%
%
% I will mention:
% \begin{itemize}
% \item How I got the s-expressions and how they were parsed
% \item The uses of the graph and the queries you can make 
% \item How much it can speed up compilation and how
% \end{itemize}
