% !TEX root =  ../Dissertation.tex

\chapter{Conclusion}


Agda Tree extracts definition dependency graphs using Andrej Bauer's
s-expression extractor \cite{andrej} and imports them into NetworkX alongside
Agda's module dependency graph. The CLI exposes queries on these graphs, that
the user utilizes to better understand the code base. The CLI is user-friendly
and easy to install for developers. Although the queries and accessibility are
somewhat limited, depending on the size of the Agda project. Queries can take a
significant amount of time and the Agda backend adds extra information that
clutters the CLI output. Agda Tree achieved the project aim to be a tool that
helps user understand large Agda codebases.

Agda Comp explores two strategies to compile modules in parallel, while
maintaining safety and correctness. The first one is level sort, which sorts
the modules into levels that can be compiled in parallel. The second being
disjoint which finds large disjoint modules that are compiled in parallel.
These two strategies were implemented into a CLI for the user to compile their
projects. Testing with these strategies using different parameters, across
different Agda libraries, showed that TypeTopology benefited the most from the
speed-up while the other libraries got modest to no improvement.

This approach is limited in two ways, one being that having to call Agda's type
checker multiple times means that the same interface files are being loaded
multiple times. The second limitation is that to create the module dependency
graph, the project is compiled which limits the use of the tool. Agda Comp
doesn't fully meet its aim to speed up compilation time, depending on the
system and project there could be no improvement.

\section{Future work}

Definition names are verbose and unintuitive, future improvements could include
a GUI to interact with the dependency graph where the user can save shorthands
for relevant definition names. The definition dependency graph also contains a
large amount of definitions that the user isn't interested in, finding an
approach to reliably remove these definitions would improve user experience.

While Agda Comp shows the potential of parallelizing type checking and the
possible time savings that could be made. In its current state it can't be used
as a drop-in replacement. To address its limitations, the Agda type checker
itself would need to be parallelised from within. This would avoid the overhead
of loading interface files for every index file compiled, all the interface
files could be loaded at once while the type checker works in parallel. Also,
to create the dependency graph the project is compiled which limits the usage
of the tool, an option to create the dependency graph without type checking
would help. Ideally this would be implemented as a part of Agda, but the source
code includes import statements for every module so generating a module
dependency graph from these import statements could be explored.

% Generally speaking, section can take different forms - as a minimum you would
% normally provide a brief summary of your project work and a discussion of
% possible future work. You may also wish to reiterate the main outcomes of your
% project and give some idea of how you think the ideas dealt with by your
% project relate to real-world situations, etc.. For the Proposal, don’t mention
% future work but do summarise your document.
%
% \begin{itemize}
% \item Compilation time could be further reduced 
% \item A clever way of removing cycles could lead to faster queries 
% \item Many useless nodes that are part of Agda's backend but not necessary for
%     the user. 
% \item Work on making the whole process more automatic and easily distributable
% \end{itemize}
