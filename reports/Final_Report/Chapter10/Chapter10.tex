% !TEX root =  ../Dissertation.tex

\chapter{Conclusion}

Agda Tree extracts definition dependency graphs using Andrej
Bauer's s-expression extractor \cite{andrej} and imports them into NetworkX
alongside Agda's module dependency graph. The CLI provides users with
queries to explore these graphs, helping understand codebase.
The tool is user-friendly and straightforward to install, making it accessible
to developers. However, its is somewhat limited by the size
of the Agda project; queries can be slow for larger projects, and additional
information from the Agda backend can clutter the CLI output. Despite these
limitations, Agda Tree achieves its main goal of helping users navigate
large Agda codebases effectively.

Agda Comp investigates two parallel compilation strategies: level sort, which
organizes modules into levels for concurrent compilation, and disjoint, which
identifies large independent modules for parallel processing. Both strategies
were implemented into a CLI tool that allows users to compile their projects.
Testing showed that TypeTopology benefited significantly from these
strategies, achieving notable speed-ups, while other libraries experienced
modest or no improvements.

Agda Comp faces two key limitations. First, calling Agda's type checker
multiple times results in redundant loading of interface files, introducing
overhead. Second, generating the module dependency graph requires compiling the
project, limiting the tool's usability to scenarios where the graph remains
unchanged. Consequently, while Agda Comp partially meets its goal of improving
compilation times, its effectiveness varies depending on the system and
project.

\section{Future work}

Definition names are verbose and unintuitive, future improvements could include
a GUI to interact with the dependency graph where the user can save shorthands
for relevant definition names. The definition dependency graph also contains a
large amount of definitions that the user isn't interested in, finding an
approach to reliably remove these definitions would improve user experience.

 In Agda Comp's current state it doesn't always improve compilation times. To
 address its limitations, the Agda type checker itself needs to be parallelised
 from within to avoid the overhead of reloading interface files. The other
 limitation could be avoided by generating the module dependency graph without
 type-checking it, this could be implemented by scraping the import statements
 from the source code.

% Generally speaking, section can take different forms - as a minimum you would
% normally provide a brief summary of your project work and a discussion of
% possible future work. You may also wish to reiterate the main outcomes of your
% project and give some idea of how you think the ideas dealt with by your
% project relate to real-world situations, etc.. For the Proposal, don’t mention
% future work but do summarise your document.
%
% \begin{itemize}
% \item Compilation time could be further reduced 
% \item A clever way of removing cycles could lead to faster queries 
% \item Many useless nodes that are part of Agda's backend but not necessary for
%     the user. 
% \item Work on making the whole process more automatic and easily distributable
% \end{itemize}
