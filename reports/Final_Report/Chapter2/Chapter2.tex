% !TEX root =  ../Dissertation.tex

\chapter{Literature Review}

% \todo[inline]{Add literature about code decay?}
% \todo[inline]{Explain what dependency graphs are?}
% \todo[inline]{Extend related works?}

Common code representation graphs include Abstract Syntax Trees (AST), Control
Flow Graphs (CFG), and Data Flow Graphs (DFG). These representations encode the
behaviour of code and help with tasks like code vulnerability detection
\cite{graph_for_code_vuln} and static analysis, where the structure is analysed
for software validation \cite{static_analysis}. For this project these graphs
encode too much information, what is needed is a dependency graph only with
definitions are their relationships.


\section{Related works}

The Language Server Protocol (LSP) is used by IDEs, such as Visual Studio Code
and IntelliJ, to provide features like \texttt{goto} definition and
\texttt{goto} references which lets the user jump to where a definition is
defined or referenced. The LSP sits between the code editor and the language
server, the language server analyses the code to support these features
\cite{LSP_implementation}. Language servers have the functionality to implement
this project, but, they are meant to be used while editing code, and the
language server isn't made to be used with custom queries. Examples of LSPs are
Jedi \cite{jedi_lsp} and Agda's language server \cite{agda_lsp}.

Ctags \cite{ctags} is a tool that indexes the symbols in a project, this is
gets the definitions from an Agda project. It helps the user with
code comprehension and gives the text editor more information to improve the
developer experience. However, it doesn't capture the relationship between the
symbols which is a key aspect of a dependency graph.

Graph Buddy is an interactive code dependency browsing and visualization tool
\cite{graph_buddy}. It takes large Java codebases and turns them into Semantic
Code Graphs (SCG), and creates a visualization of this graph. It shows
dependencies between modules, classes, and methods which helps the developer
better understand the project and tackle the pervasive issue of code decay
\cite{code_decay_evidence}. Graph Buddy is integrated into an IDE as a plugin,
where the user can seamlessly explore the visualization. This is quite similar
to the goal of this project, it contains all the definitions and their
relationships. Furthermore, the visualization also makes simpler to use
than a CLI. However, it is exclusive to Java.

There is also a tool \cite{call_graph_vis} that visualizes the call graph of a
coding project, this allows for better developer experience. It works in
real-time, while the user is editing the code the graph will automatically
update to show the changes. Helping with developer experience and understanding
how the code flows through the project. This isn't relevant to Agda, as a proof
assistant it helps with proving theorems which doesn't fit with call graphs.

\section{Parallel Compilation}

% \todo[inline]{Add section explaining DAGs?}

\todo[inline]{Is there any related work that I could add?}

Type-checking is a computationally expensive sequential task that hinders
workflow during development, so parallelizing could lead to a speed-up like in
other applications. Parallel type-checking aims to type-check different parts
of a project at the same time, while incremental type-checking aims to allow
the developer to type-check a small change in the project without having to
type-check the whole project again. Both can reduce compilation time, improving
the developer experience.

An example is the work by Newton et al. \cite{paralele_comp_haskell} which
seeks to parallelise type-checking with Haskell. Also, the work by Zwaan et al.
\cite{incremental_type_checking} using scope graphs for incremental
type-checking. These tools give methods to improve type checkers for general
applications, there isn't a focus on Agda.

This project doesn't aim to optimise the type-checking algorithm
itself, rather, find independent modules that can be type checked together.
This aligns more closely with the following paper that explores reducing FPGA
compile time by changing from a monolithic compilation style to compiling
separate blocks in parallel \cite{FPGA}.

\todo[inline]{Add section explaining scheduling problem?}

Fundamentally the compilation problem in this project is a scheduling problem,
which is NP-complete\cite{scheduling}. This means that there many algorithms
that attempt to tackle this problem by following different assumptions as shown
by Yves Robert \cite{scheduling}.

\section{MLFMF: Data Sets for Machine Learning for Mathematical Formalization}

MLFMF is a collection of data sets for benchmarking systems that help
mathematicians find relevant theorems when proving a new one
\cite{bauer2023mlfmf}. The data sets are created from large libraries of
formalized mathematics in Agda and Lean. They represent each library as a graph
and as a list of s-expressions. Some of the libraries included are as
Agda-unimath and TypeTopology. The collection is a base for investigating
machine learning methods to mathematics. The methodology to extract
s-expressions can be used with other libraries to continue expanding the 250000
entries in the data sets.

The s-expressions extractor is created in Agda by extending the backend
\cite{andrej}. The backend has access to internal information about a project's
definitions and its connection with other theorems. Andrej Bauer used that
backend and converted the internal information into easy-to-parse
s-expressions. The Figure \cref{fig:example-sexp} is an example of the
\texttt{:entry} s-expression, each \texttt{:entry} tag marks where a theorem is
defined, and it contains the name of the theorem, its type, and its
implementation. Mind that \texttt{(...)} are more s-expressions that were
replaced for readability. A more general structure to the tag can be seen in
\cref{fig:sexp-ast} where the three parts are shown as subtrees and a node.

\begin{figure}[H]
  \begin{subfigure}[b]{0.40\textwidth}
    \centering
    \begin{tabular}{p{4cm}}
      \texttt{(:entry}\newline
      \hphantom{oo}\texttt{(:name} $\mathbb{N}$\texttt{)}\newline
      \hphantom{oo}\texttt{(:type (...))}\newline
      \hphantom{oo}\texttt{(:data}\newline
      \hphantom{oooo}\texttt{(...)}\newline
      \hphantom{oooo}\texttt{(:name} $\mathbb{N}$\texttt{.zero)}\newline
      \hphantom{oooo}\texttt{(:name} $\mathbb{N}$\texttt{.suc)}\newline
      \hphantom{oo}\texttt{)}\newline
      \texttt{)}
    \end{tabular}
    \caption{\raggedright Example s-expression from MLFMF Figure 1(b) \cite{bauer2023mlfmf}.}
    \label{fig:example-sexp}
  \end{subfigure} \hfill
  \begin{subfigure}[b]{0.55\textwidth}
    \centering
    \begin{tikzpicture}
      \Vertex[x=0, size=0.8, label={Entry}, RGB, color={128,200,128}]{e}
      \Vertex[x=-2, y=-0.8, size=0.8, label={name}, RGB, color={75,175,128}]{n}
      \Vertex[x=0, y=-2, size=5.2, label={\tiny declaration}, RGB, color={75,175,128}, style={isosceles triangle, minimum width=1cm, rotate=90, minimum height=2cm}]{d}
      \Vertex[x=2.25, y=-0.6, style={draw=none}, RGB, color={255, 255, 255}]{fake}
      \Vertex[x=2, y=-2, size=1, label={body}, RGB, color={75,175,128}, style={isosceles triangle, minimum width=1cm, rotate=90, minimum height=2cm}]{b}
      \Edge[](e)(n)
      \Edge[](e)(d)
      \Edge[](e)(fake)
    \end{tikzpicture}
    \caption{Graph representing s-expression entry tag containing a name,
    declaration and body from MLFMF Figure 3 \cite{bauer2023mlfmf}.}
    \label{fig:sexp-ast}
  \end{subfigure}
\end{figure}

The \texttt{:entry} tag is of important, as Agda by itself doesn't provide a
convenient method to find all the definitions of an Agda project along with its
implementation. This extractor packages the important information into a format
that is easy to parse. Andrej Bauer made the s-expression
extractor open-source in a GitHub repository \cite{andrej}.

This paper also describes another graph that can be generated from these
s-expressions, this graph contains all the information about each theorem and
what definitions it uses and how. While this graph could be queried by the
user, it contains a too much information that is unnecessary for the
user to explore.

\section{Conclusion}

There are many tools that analyse and visualize the structure of programming
projects. They are used in static code analysis, where software can be
validated and developers can use them to explore a project. However, most of
the tools are not easy to query by the user, are meant to be used while editing
a file or are reserved for more popular languages like Java. A tool that can
read an Agda project and give access to the underlying graph is missing.

The MLFMF paper \cite{bauer2023mlfmf} describes a methodology to extract the
definitions and dependencies of an Agda project into s-expression. Which are
easier to parse than Agda source code. Which is then converted into a graph
that contains all the details of the Agda source code.

Slow compilers are a common problem which hurts developer experience. Due to
the monolithic nature of type-checker, parallelization becomes a route to
reduce compilation time. While parallelization can be applied to the
type-checking algorithm itself, this project looks to type-check modules in
parallel while the type-checking algorithm remains the same. This problem is
closer to a scheduling problem, where the goal is to find the optimal way to
assign tasks to multiple machines to reduce completion time. No tool in Agda
attempts to apply a scheduling algorithm to the type-checking of modules, which
could lead to significant speed-ups.  


% This is often referred to the ‘literature review’ section. It is one of the most important
% section of the Project Proposal and the Project Report. It is where you demonstrate that you
% understand the state-of-the-art in the field you’re working. Towards the end of this sections it’s
% normally a good idea to explain how your aim / work / idea / contribution differs from the nearest
% work in the field.


% Research papers to explore:
% \begin{itemize}
%     \item State of the art in dependency graph extraction 
%     \item state of the art in dependency graph exploration
%     \item State of the art in Agda compilation 
% \end{itemize}
%
% \begin{itemize}
% \item s-expressions 
% \item How graphs can be used to represent the definitions
% \item Most tools focus on data-flow for static analysis, I am looking
%     for a simpler tools that allows the querying of the relationships between
%     definitions.
% \end{itemize}
%
% Tools that do a similar job
% \begin{itemize}
% \item \href{https://github.com/universal-ctags/ctags}{CTags}
%     \begin{itemize}
%     \item CTags indexes all the symbols of a project
%     \item This works only on definitions of functions, not where they are referenced
%     \end{itemize}
% \item \href{https://jedi.readthedocs.io/}{Jedi - an awesome autocompletion, static analysis and refactoring library for Python}
%     \begin{itemize}
%     \item It finds function definitions and references which is what we are looking
%         for
%     \end{itemize}
% \item \href{https://github.com/agda/agda-language-server/tree/master}{Agda-Language-Server}
%     \begin{itemize}
%     \item An LSP for agda, it parses the files and analysis them, similar to what we
%         want but it doesn't seem to index them
%     \end{itemize}
% \item \href{https://cormack.uwaterloo.ca/~olhotak/pubs/ecoop12.pdf}{Application-only Call Graph Construction }
%     \begin{itemize}
%     \item A call graph represents the relationships between functions
%     \end{itemize}
% \item \href{https://dl.acm.org/doi/pdf/10.1145/199448.199462}{Precise Interprocedural Dataflow Analysis via Graph Reachabilit}
%     \begin{itemize}
%     \item Data flow is a way to statically analyse code before running it, this is
%         far more complex than what I want.
%     \end{itemize}
% \item \href{https://dl.acm.org/doi/abs/10.1145/3583660.3583664}{CFL/Dyck Reachability: An Algorithmic Perspective}
%     \begin{itemize}
%     \item Analyses the decidability and complexity of this problem
%     \end{itemize}
% \end{itemize}

