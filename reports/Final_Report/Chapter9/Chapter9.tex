% !TEX root =  ../Dissertation.tex

\chapter{Evaluation}

Usually you evaluate your project with regard to the functional and non-functional
requirements you set out in the earlier chapter. This doesn’t necessary mean that your project was
successful but, if these requirements were appropriately specified, you it’s likely that your project was
successful. You might be reiterating some points from the Testing and Success Measurement
chapter in your Project Report.

\begin{itemize}
\item Mention compilation time reduction, and improvements 
\item Mention how some queries take too long but were all implemented 
\item Mention how trees take some time but can be re-used instantaneously 
\item Creates graph from any agda project
\end{itemize}

\begin{itemize}
\item Explain how agda tree meets the functional and non-functional requirements.
\item Discuss limitation 
\item where the tool might need improvement 
\item Recap of how the CLI tool meets the research objectives 
\item integraiont into ides
\end{itemize}

\begin{itemize}
\item Explain how the strategies are giong to be tested 
\item Explain what strategies will be tested and why 
\item Explain what libraries will be used for testing 
\item Show table with the results of each test
\item Discuss limitation, constrints of scenarios 
\item where the tool might need improvement 
\item Recap of how the CLI tool meets the research objectives
\end{itemize}

